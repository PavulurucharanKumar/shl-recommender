\documentclass[a4paper,11pt]{article}
\usepackage{geometry}
\geometry{margin=1in}
\usepackage{parskip}
\usepackage{enumitem}
\usepackage{amsmath}
\usepackage[T1]{fontenc}
\usepackage{noto}

\begin{document}

\title{SHL Assessment Recommendation System Approach}
\author{}
\date{}
\maketitle

\section*{Overview}
The SHL Assessment Recommendation System recommends up to 10 SHL assessments based on a job description or query, respecting duration constraints. It uses NLP for matching, with a Streamlit UI, Flask API, and evaluation metrics.

\section*{Approach}
\begin{itemize}
    \item \textbf{Data}: Stored assessments in JSON with name, URL, description, duration, remote/adaptive support, test type.
    \item \textbf{Engine}: Used Sentence-BERT to rank assessments by cosine similarity, filtering by duration.
    \item \textbf{App}: Built a Streamlit UI and Flask API with \texttt{/health} and \texttt{/recommend} endpoints.
    \item \textbf{Evaluation}: Computed Mean Recall@3 (0.50) and MAP@3 (0.33) on test data.
\end{itemize}

\section*{Tools}
Python, Sentence-Transformers, Flask, Streamlit, Pandas, NumPy, Scikit-learn, Render.

\section*{Optimization}
Improved scores by tuning similarity thresholds.

\section*{Evaluation}
Used test dataset to compare predictions against ground truth, calculating Recall@3 and MAP@3.

\end{document}